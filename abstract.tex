\begin{abstract}
\vb{I have made many small changes to the abstract. Please proof-read. Also we should update it on the AAAI submission website.}

Crosswords puzzles continue to be a popular form of entertainment.
In Artificial Intelligence (AI), crosswords can be represented as a constraint problem,
and attacked with a combinatorial search algorithm.
In combinatorial search,
the branching factor can play a crucial role on the search space size
and thus on the search effort.
We introduce tiered state expansion, a completeness-preserving technique
to reduce the branching factor.
In problems where the successors of a state correspond to the 
values in the domain of a state variable selected for instantiation,
the domain is partitioned into two subsets called tiers.
The instantiation of the two tiers is performed at different times,
with tier 1 first and tier 2 
in a subsequent state.
Before a tier-2 instantiation occurs,
its set of applicable values can shrink
substantially due to constraint propagation,
with a corresponding reduction of the branching factor.
We apply tiered state expansion to a constraint optimization problem
modeled on the Romanian Crosswords Competition, empirically demonstrating a substantial improvement.
Tiered state expansion allows finding a full solution,
with an average CPU time of up to $1.2$ minutes,
to many puzzles that would otherwise time out after 4 hours.
\end{abstract}


\section{Background}
\label{sec::background}

In this section we discuss how standard crosswords grid generation and a few important generalizations
can be mapped into constraint satisfaction and constraint optimization problems.

\subsection{Constraint Satisfaction and Constraint Optimization}

%We address the same problem as~\citeauthor{DBLP:conf/socs/BoteaB21}~\shortcite{DBLP:conf/socs/BoteaB21}.
We consider constraint optimization problems
obtained from
constraint satisfaction problems (CSPs) by adding a score function on partial
or full solutions.

\bdf
A \emph{constraint satisfaction problem (CSP)} is a tuple ${\cal P} = \langle X, D, C \rangle$,
where:
\begin{itemize} 
\item $X = \{v_1, \dots, v_n\}$ is a collection of variables;
\item $D = \{D_1, \dots, D_n\}$ is a collection of finite domains with $D_i$ corresponding to variable $v_i$;
\item $C = \{C_1, \dots, C_m\}$ is a collection of constraints. Each constraint $C_j$ is a pair $\langle t_j, R_j \rangle$, where $t_j \subset X$ is a subset of $p$ variables and $R_j$ is a $p$-ary relation on the corresponding subset of domains.
\end{itemize}
\edf

\noindent We define $D^*$ as an extension of $D$ that contains a special symbol $\bot$ in each domain for variables with no actual value assigned yet. $D^*$ can be used to represent partial assignments.

\bdf
An \emph{optimization CSP} is a tuple $\langle {\cal P}, f \rangle$, 
where ${\cal P}$ is a CSP and 
$f : D^* \rightarrow \mathbb{R}^+$ is a score function.
\edf

An assignment $s \in D^*$ is {\em consistent} if 
it violates no constraint.
A consistent partial assignment is called a {\em partial solution}.
A consistent full assignment is a {\em full solution} (shorter, solution).
A solution is optimal if no solution has a higher score.

\subsection{Crosswords Grid Generation}

Crosswords grid generation is a well-known, textbook example of a CSP.
In a crosswords grid generation instance, the input is a list of words, and 
a grid with black cells and white cells.
Define a word slot as a sequence of contiguous cells, either on a row or a column,
bordered at each end of the sequence with either a black cell or the border of the grid.
A popular way to model crosswords grid generation as a CSP is to 
define one CSP variable for each word slot.
Its domain is the list of words of the corresponding length.
The crosswords grid generation problem is NP-complete \cite{garey1979computers,10.1007/978-3-642-30347-0_15}.

\subsection{Optimization Crosswords}

Optimization crosswords introduce a scoring function on top of the 
standard crosswords grid generation.
Similarly to crosswords grid generation, the problem is NP-complete \cite{DBLP:conf/socs/BoteaB21}.
However, in practice, optimization crosswords instances often are much more
challenging compared to standard grid generation.
Intuitively, this is because in standard grid generation any correct solution will do,
whereas in optimization crosswords we are interested in solutions with a high score.

\subsection{Crosswords with Dynamic Black Cells}

In standard grid generation problems, the placement of black cells is
given as input.
In contrast, in crosswords with dynamic black cells, the initial grid contains 
only white cells. 
The task is to fill the grid with a combination of black cells
and criss-crossing words from the input list.
Black cells added to the grid need to satisfy a number of constraints,
depending on the exact problem formulation.
A maximum number of black cells (as a percentage of the total number of cells)
is a typical example of such a constraint.
The next section shows in detail the constraints related to black cells 
that we consider in this work.

The need to add black cells as part of the solving process has two 
major implications.
Firstly, the state space blows up combinatorially.
For instance, on a $13 \times 13$ grid with $26$
black cells, the total number of combinations is $169 \choose 26$,
not all of which are legal, according to the constraints
specified in the next section.

Secondly, adding black cells dynamically has the important consequence that
word slots are introduced gradually.
The number of word slots is initially zero, and it grows
as more and more black cells are placed on the grid.
In other words, the CSP variables of the problem instance are introduced gradually,
rather than being all defined from the beginning.
Before all black cells are placed on the grid, 
the constraint problem modelling has only a \emph{partial} problem definition,
making the solving approach more challenging.

As mentioned in the introduction, and also seen in the next section, the {\sc Roco} problem features both 
an optimization component and dynamic black cells, on top of the
standard crosswords grid generation problem,
inheriting a combination of 
the computational challenges outlined earlier in this section.
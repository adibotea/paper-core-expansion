\section{Empirical Evaluation}


\begin{figure*}[th]
\centering
%\includegraphics[width=0.56\textwidth]{_empiricalSupport/_originalSubmission/y-2013/results/_runWombat/mrmeGrids_y2013-60x932-14400x352-14113466-paper.pdf}

%\vspace{0.15cm}

%\includegraphics[width=0.56\textwidth]{_empiricalSupport/_originalSubmission/y-2021/results/_runWombat/mrmeGrids_feb3-60x38035-14400x224-13741779-paper.pdf}

%\vspace{0.15cm}

%\includegraphics[width=0.56\textwidth]{_empiricalSupport/_originalSubmission/y-2023/results/_runWombat/mrmeGrids_y2023-60x446-14400x352-14179462-paper.pdf}


\includegraphics[height=5cm]{2013a.png}
%
\hspace{1cm}
%
\includegraphics[height=5cm]{2021a.png}
%
\hspace{1cm}
%
\includegraphics[height=5cm]{2023a.png}

\vspace{0.25cm}

\includegraphics[height=5cm]{2013b.png}
%
\hspace{1cm}
%
\includegraphics[height=5cm]{2021b.png}
%
\hspace{1cm}
%
\includegraphics[height=5cm]{2023b.png}

\caption{Our top results for years 2013 (left column), 2021 (middle column) and 2023 (right column). Top: seeds after placing additional black cells. Bottom: {\sc Wombat} solutions.}
\label{fig:results}
\end{figure*}

We applied the approach presented above to synthesize grids for three years of the competition: 2013, 2021 and 2023. Each year uses its own thematic dictionary. Note that competition results and human-designed grids for year 2023 are not yet announced. Our code and data are available at \url{https://www.dropbox.com/s/d1g9l7vthpw35sh/code-data-socs23.tgz?dl=0}.


For each of the three years we generated and expanded cores (lines~\ref{algl:generateCores} and \ref{algl:expandCores} in Algorithm~\ref{algo:core-exp}), and further generated seeds (line~\ref{algl:generateSeeds}) that were 
post-processed and then ranked (line~\ref{algl:rankSeeds}). Top-ranked seeds were then evolved (line~\ref{algl:evolveSeeds}). Summary statistics about such steps are listed in Table~\ref{tab:rankingEvolving}.

The ranking and evolution steps were repeated four times for the same seed set for each year. Each run used a different random-number generator initialization. As per Section~\ref{sec:evolution}, each ranking + evolution run took about two days on a $32$-core CPU. We used four $32$-core CPU nodes on a grid and thus were able to execute all four runs in parallel. Out of the four runs the highest score grids synthesized for each year are shown in Figure~\ref{fig:results}. On the left we show evolved seeds with the contents of the original seed shown in tinted background. Additional black cells added during the evolution are shown without tinting.

The human competition publishes the top 12 entries every year. Years 2013 and 2021 feature in their top-12 entries at least one entry that contains the $6 \times 4$ pattern considered in our experiments (Figure \ref{fig:pattern}, left). In 2013 scores in the top 12 list ranged from $184$ (first place) to $181$ (second place) and so on down to $175$ on the twelfth place. Our best score out of the four runs for that year is $182$ with the mean and standard deviation of $4 \cdot 352$ scores of $159.6  \pm 20.83$. 
One human entry from 2013 features the $6 \times 4$ pattern mentioned. 
Our solution has the pattern in a different position, as well as part of the letters different inside the pattern.
This further leads to significant differences between these two solutions.
In 2021 human scores in the top-12 list range from $195$ down to $187$ whereas our top score out of the four runs is $190$ ($162.6 \pm 25.02$). Two top-12 entries in 2021 feature the perfect-score pattern considered in our work. One has $194$ points and one has $189$ points. Our best entry from the four runs has $190$ points and turns out to be similar to the $194$-point entry. In particular, the core has the same letters in both cases and the same position in the grid. In contrast, the $189$-point human score differs significantly from our solution. In particular, the $6 \times 4$ pattern is in a different position and the letters inside the pattern differ in part. Year 2023 human results are not announced yet. Our top score of the four runs is $186$ with the mean and standard deviation being $147.9 \pm 36.11$.\footnote{While it was previously known that starting from an empty grid tends to result in scores of evolved grids below that of champion-level human-created grids, we did run some experiments with MAP-Elites. The empirical setup was somewhat different from the that used for the main experiments yet the scores of grids MAP-Elites synthesized from an empty grid were indeed lower than those when starting with seeds: $154$ versus $190$ for year 2021 and $141$ versus $186$ for year 2023.}

Thus with additional input as simple as a pattern to use which in our experiments is a $6 \times 4$ rectangle with two black cells in its corners, our approach can reach scores that are competitive with those from top human performers. Part of the previous work utilized much more detailed additional input such as the entire configuration of black cells~\cite{DBLP:conf/socs/BoteaB21}. When no such input is used, the best results reported were significantly behind top-12 results in that
year~\cite{DBLP:conf/cig/BulitkoB21,Botea_Bulitko_2022}.


\section{Romanian Crosswords Competition}
\label{sec::roco}

%\ab{Still an almost verbatim copy of the last paper. Todo: edit.}

In this section we describe the application domain in detail,
for a self-contained paper.
The Competition is an annual contest with human participants, started in 1965.
%Until 2021, it has been organized by {\em Rebus}, a central Romanian publication
%dedicated to crosswords.
The task is to create a $13\times13$ grid filled with words and
at most $26$ black cells.
The input includes two lists of words (dictionaries), called the \emph{thematic list}
and the \emph{regular list}.
The thematic list changes every year and is on the order of a few hundred words.
The regular list contains approximately $135$ thousand Romanian words.
If the two lists overlap, the common words 
are considered to be thematic, and removed from the regular list.
Black cells cannot have common edges but they are allowed to have common corners. 
White cells have to form a cardinally connected region. Black cells cannot create semiclosures --- configurations of black cells such that
adding one more black cell would partition the open/white area of the grid into two or more
areas with the disconnected areas having more than one open cell each.

%\begin{definition}
Recall that a {word slot}
is a set of contiguous white cells in a row or a column
where each end of the slot is adjacent to either the 
border of the grid or a black cell.
%\end{definition}
%
Slots of length $1$ can be filled with any letter.
Slots of length $2$ can be filled with any combination of two letters
but no two-letter combination can be repeated in the grid.
Slots of length $3$ or higher can be filled with words from the two lists.
For simplicity we say that singleton letters are words
of length $1$ and two-letter combinations are words of length $2$.
We treat them as regular words unless such a combination is in the thematic list.% which is uncommon.

Words of length $2$ or higher cannot be repeated in a grid.
So-called families of words are forbidden (e.g., 
{\sf\small WRITER} and {\sf\small WRITING} cannot both be placed on the same grid).\footnote{We do not observe the family-of-words constraint as no mapping of words into their families is available to us.
We manually inspected our 
solutions reported in this paper, confirming that our solutions satisfy this constraint as well.}
All cells in a (full) solution must contain a letter or a black cell.
%Two intersecting slots must agree on the letter placed in the common cell.

The score of a grid is the sum of the lengths of all thematic words included in the solution. 
This is equivalent to saying that, in a solution each cell at the intersection of two
thematic words contributes two points. A cell at the intersection of one thematic word
and one regular word contributes one point. A cell at the intersection of two 
regular points contributes zero points, and so does every black cell.

%For example, in the nine-year dataset that we use in our study in this work,
%the thematic dictionary varies from $278$ to $481$ words,
%with the median size being $387$. 
%Using a word from the thematic dictionary gives a point for every letter.
%Thus, every cell in the grid can give at most two points, when it lies at the
%intersection of two thematic words.

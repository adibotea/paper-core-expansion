\section{Introduction}
\label{sec::intro}

Many instances of NP-hard constraint optimization problems
remain challenging to current technology.
In such problems, the task is to find a solution
(e.g., assignment to variables that define a state) 
that satisfies a number of constraints specified in the definition
of the problem, while optimizing an objective function
(such as maximizing a score or minimizing a cost).


We present an approach that first builds a \emph{core},
a part of the problem that will have a strong contribution
to the score (objective function) of the solution.
A core is utilized to build a partial solution called a seed.
A seed has part of the state variables specified and instantiated.
A seed is further evolved into a full solution, in a way that aims
at maximizing the overall score.

We apply our ideas to the Romanian Crosswords Competition Problem (shorter, {\sc Roco}),
introduced in the search literature by
\citeauthor{DBLP:conf/socs/BoteaB21}~\shortcite{DBLP:conf/socs/BoteaB21}.
The application has a decades-long history of annual national-level competitions.
The problem is challenging to AI, which has been lagging significantly
behind the performance of top human contestants,
despite recent progress~\cite{DBLP:conf/socs/BoteaB21,DBLP:conf/cig/BulitkoB21,Botea_Bulitko_2022}.

In {\sc Roco}, the input includes two lists of words (the thematic list and the regular list),
and a $13 \times 13$ grid with white cells.
The task is to fill the grid with criss-crossing words and no more than 26 black cells.
Each thematic word gives a number of points equal to its length. The objective
is to obtain as many points as possible.

{\sc Roco} generalizes a well-known problem that we
call the \emph{standard crosswords grid generation}.
The latter is a textbook example of a constraint satisfaction problem.
The input is a list of words, and a grid with black cells and white cells.
The task is to fill the white-cell area with criss-crossing words.

{\sc Roco} generalizes the standard crosswords grid generation in two ways,
and the resulting differences amplify the computational difficulty of {\sc Roco}.
Firstly, in {\sc Roco}, solutions are ranked by their score,
whereas in the standard problem any correct solution will do.
Often, finding a high-quality solution is much more challenging than finding any solution.
Secondly, the task in {\sc Roco} includes finding a configuration of black cells
(that will allow achieving a high score), whereas in standard grid
generation the configuration of black cells is given as input.
The absence of a pre-existing black cell configuration
blows up the search space combinatorially, and good configurations
(i.e., that allow achieving a high score) are difficult to find.

%Recent work in {\sc Roco} showed that state-of-the-art AI significantly lags behind
%top human performance, despite a significant recent progress.

Our approach takes as input the size of a rectangular core,
and the locations of zero or more black cells inside the core.
We report a substantial improvement of the scores achieved compared
to previous work, in a computational time that is orders of magnitude shorter.
Our system is capable of generating scores in the vicinity of human champion scores.
We believe that our work is a breakthrough on the journey to exceeding
human performance in this challenging domain.



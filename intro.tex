\section{Introduction}
\label{sec::intro}

\ab{Started editing the introduction.
Vadim, please don't proofread yet, as it's not a finished draft yet.}

Constraint optimization 
is a challenging combinatorial search problem.
A broad class of constraint optimization problems can be modelled utilizing
a finite set of variables with a finite discrete domain each. 
A state corresponds to an instantiation of zero or more variables. 
%Successors of a state are defined by selecting an 
%uninstantiated variable and instantiating it with some/every of its possible values.
%Each such instantiation produces as a successor state.
Legal states need to satisfy a set of constraints specified in the definition of the problem instance.
An objective function maps states to a solution cost (lower is better) or score (higher is better).
Without loss of generality, in this work with focus on maximizing a score.

We present an approach that first builds a \emph{core} of a solution,
an instantiation to a subset of variables that have a strong contribution
to the score function.
A promising core is then extended into a full solution.

We apply our ideas to the Romanian Crosswords Competition Problem (shorter, {\sc Roco}),
introduced in the search literature by
\citeauthor{DBLP:conf/socs/BoteaB21}~\shortcite{DBLP:conf/socs/BoteaB21}.
The application has a decades-long history of annual national-level competitions.
The problem is challenging to AI, which has been lagging significantly
behind the performance of top human contestants.

In {\sc Roco}, the input includes two lists of words (the thematic list and the regular list),
and a $13 \times 13$ grid with white cells.
The task is to fill the grid with words and no more than 26 black cells.
Each thematic word gives a number of points equal to its length. The objective
is to obtain as many points as possible.

{\sc Roco} generalizes a well-known problem that we
call the \emph{standard crosswords grid generation}.
The latter is a textbook example of a constraint satisfaction problem.
The input is a list of words, and a grid with black cells and white cells.
A contiguous sequence of white cells (either horizontal or vertical), bordered at 
each end by either a black cell or the grid border, forms a so-called word slot.
The task is to fill each word slot with a word from the dictionary, so that
intersecting words match (i.e., have the same letter) on their common position.
Additional constraints might be imposed, such as avoiding word repetition in a solution.

{\sc Roco} generalizes the standard crosswords grid generation in two ways.
First off, in {\sc Roco}, solutions are ranked by their score,
whereas in the standard problem any correct solution will do.

Second off, in 
standard crosswords grid generation, black cells are given apriori as part of the input.
In contrast, in the {\sc Roco} problem,
black cells need to be placed as part of the solution generation process.

This difference has two important implications.
The first implication is that the state space blows up in {\sc Roco}. 
For example, on a $13 \time 13$ grid with $26$ black cells,
the number of combinations is in the order of $169 \choose 26$.

The second implication is that placing black cells dynamically significantly impacts 
the formulation of the problem as a constraint optimization problem, making 
{\sc Roco} different from typical constraint optimization problem formulations.
Typically, constraint optimization problems have their set of variables fixed
and defined from the beginning.
In particular, this is the case with the standard grid generation problem,
where the placement of black cells is given from the beginning:
all word slots are defined at the beginning and each word slot defines one variable.

On the other hand, in {\sc Roco}, word slots are defined gradually, 
as more and more black cells are being added on the grid.
This implies that the formulation as a constraint optimization problem
starts with an empty set of variables, and the set of variables grows gradually.


Recent work in {\sc Roco} showed that state-of-the-art AI significantly lags behind
top human performance, despite a significant recent progress.

Part of recent efforts focus on an easier subproblem, assuming that
a good placement of black cells is given apriori as part of the input
~\cite{DBLP:conf/socs/BoteaB21,Botea_Bulitko_2022}.
By decomposing the problem in such a way, researchers have been capable
to bring the word-placing component to a strong level,
where the scores of human champions can effectively be reproduced (and in some
cases even exceeded) if the black cell configuration of human-champion grids
is given as input.
However, requiring the black cell configuration as input remains a major limitation
of such previous work. Different black cell configurations vary greatly in quality
(defined as the maximum score that can be achieved
starting from a given black cell configuration),
and good-quality black cell configurations are very difficult to find.

Other recent work has focused on generating black cell configurations
automatically. The performance remained significantly beyond
the top human performance.

In this work we generate black cells automatically.
We require as input a so-called pattern, which is a
rectangular area with zero or a few black cells (e.g., two black cells).
We reach full solutions with scores that would place us among the top 12
results.\footnote{The top 12 results are considered as distinguished
results in the competition, with the corresponding solutions published.
All other competitors get their score published, but not the full solution.}

Compared to Blah, our work requires only a pattern as additional input,
as compared to the entire configuration of black cells.
Compared to Blah, we achieve top scores (e.g., 190 vs 182), and in a much shorter time
than Blah required to reach 182 points.

We believe that our results are a breakthrough on the journey to eventually
exceeding best human performance with fully automated solutions.
Reaching such a point further requires: an automated selection of the input pattern,
and further pushing scores up from a top-12 level to higher than the best human score.

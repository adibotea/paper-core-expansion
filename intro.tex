\section{Introduction}
\label{sec::intro}

\vb{I have made many changes through the section. Please proof-read.}

Crosswords puzzles are a very popular form of entertainment.
We address a problem
inspired by the Romanian Crosswords Competition and introduced in the search literature by
\citeauthor{DBLP:conf/socs/BoteaB21}~\shortcite{DBLP:conf/socs/BoteaB21}.
The task is to fill a crosswords grid with words and black cells.
Some words, called thematic words, give points. The objective
is to obtain as many points as possible.
The application has a decades-long history of annual national-level competitions.
The problem is challenging to AI, which lags 
behind the performance of top human contestants.
%
The problem can be represented as a constraint optimization and tackled with a search algorithm. As usual in a search, the branching factor
(i.e., the number of successor states to a parent state)
can greatly impact the search space size and search time.

We present {\em tiered state expansion}, an approach to reducing the branching factor.
We adopt a few assumptions that characterize 
an important and broad class of constraint optimization problems.
Specifically our states are defined by a finite set of variables with a finite discrete domain each. A  state corresponds to an instantiation of zero or more variables. Successors of a state are defined by selecting an uninstantiated variable and instantiating it with some/every of its possible values.
%Each such instantiation produces as a successor state.
Legal states need to satisfy a set of constraints specified in the definition of the problem instance.
An objective function maps states to a solution cost (lower is better) or score (higher is better).
Constraint propagation can gradually shrink domains of uninstantiated variables based on  
values of already instantiated variables.

For example, in a crosswords application, variables can correspond to slots,
where a slot is defined as a horizontal or vertical sequence of white cells bordered
at each end by either the margin of the grid or by a black cell.
The domain of such a variable is the set of words that fit into that slot.
At a given point in time, a slot (variable) can be instantiated with a word (value) or uninstantiated.
A state comprises all slots with their corresponding assignment at that time (i.e., either 
filled with a word or uninstantiated).

Tiered state expansion partitions the current domain of a variable selected for instantiation
into two subsets called {\em tier 1} and {\em tier 2}.
The two tiers are defined so that a tier-1 subset is likely to contain values that could
lead to higher-quality solutions, and a tier-1 subset is preferably significantly smaller than the current domain before splitting.
When the variable is selected for expansion for the first time,
only tier-1 values are used to generate successor states.
An additional, special successor state called a \emph{placeholder successor}
leaves the variable uninstantiated but flags it so that any future instantiations
(i.e., in a state down in the subtree of the placeholder successor)
will use only tier-2 values.
Thus, tier-1 and tier-2 instantiations 
for a given variable are separated during the search
and other variable instantiations can occur between those two.
By the time tier-2 instantiations would be performed,
the number of remaining tier-2 values can be much smaller
due to constraint propagation.
This can result in a significant reduction of the actual
branching factor.
%\footnote{\citeauthor{DBLP:conf/socs/BoteaB21}~\shortcite{DBLP:conf/socs/BoteaB21}
%mention the tiered expansion idea very briefly, in passing, whereas this paper
%presents and evaluates the idea in depth.}

Our algorithmic contributions are applicable to constraint optimization problems that match
the assumptions outlined earlier in this section.
We require only partitioning a variable domain into two tiers,
with tier 1 preferably small, 
and with 
tier-1 likely to contain some values that would lead to high-quality solutions.

Our ideas are easy to understand and implement.
They can be implemented as extensions to a search algorithm,
or as new constraints in a constraint satisfaction problem.
For simplicity, we adopt the former approach.

Tiered state expansion is implemented in {\sc Wombat}, an existing state-of-the-art crossword puzzle solver~\cite{DBLP:conf/socs/BoteaB21}.
Experiments demonstrate substantial speed and memory gains. Consequently, tiered state expansion allows to fully solve puzzle instances that,
with standard expansion, would fail after a four-hour search.
We make our code and experiments available to
accelerate progress
% and exceeding human performance
in this challenging problem.

\begin{comment}
The rest of the paper is organized as follows.
First we review related work, followed by the problem formulation.
Then we present the Romanian Crosswords Competition,
followed by a background on constraint optimization as a search problem.
We present the tiered state expansion in detail,
and evaluate it on Romanian Crosswords puzzle instances.
We end with future work and concluding remarks.
\end{comment}
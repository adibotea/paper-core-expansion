\section{Problem Definition}
\label{sec::problem}

\ab{Still a verbatim copy of the last paper. Todo: edit.}

We address the same problem as~\citeauthor{DBLP:conf/socs/BoteaB21}~\shortcite{DBLP:conf/socs/BoteaB21}.
It is modeled as a type of constraint optimization obtained from a
constraint satisfaction problem (CSP) by adding a score function on partial
or full solutions.

\bdf
A \emph{constraint satisfaction problem (CSP)} is a tuple ${\cal P} = \langle X, D, C \rangle$,
%\vb{Sets do not allow repetitions and do not have order. If $X,D,C$ are really sets then all $D_i$ needs to be distinct which is not the case for us since a domain of any $n$-letter word slot is the same at the beginning of the search. Perhaps using the term ``collection'' instead of ``set'' would fix this?}
where:
\begin{itemize} 
\item $X = \{v_1, \dots, v_n\}$ is a collection of variables;
\item $D = \{D_1, \dots, D_n\}$ is a collection of finite domains with $D_i$ corresponding to variable $v_i$;
\item $C = \{C_1, \dots, C_m\}$ is a collection of constraints. Each constraint $C_j$ is a pair $\langle t_j, R_j \rangle$, where $t_j \subset X$ is a subset of $p$ variables and $R_j$ is a $p$-ary relation on the corresponding subset of domains.
\end{itemize}
\edf

\noindent We define $D^*$ as an extension of $D$ that contains a special symbol $\bot$ in each domain for variables with no actual value assigned yet. $D^*$ can be used to represent partial assignments.

\bdf
An \emph{optimization CSP} is a tuple $\langle {\cal P}, f \rangle$, 
where ${\cal P}$ is a CSP and 
$f : D^* \rightarrow \mathbb{R}^+$ is a score function.
\edf

An assignment $s \in D^*$ is {\em consistent} if 
it violates no constraint.
A consistent partial assignment is called a {\em partial solution}.
A consistent full assignment is a {\em full solution} (shorter, solution).
A solution is optimal if no solution has a higher score.
